% AER-Article.tex for AEA last revised 22 June 2011
\documentclass[AER]{AEA}

% The mathtime package uses a Times font instead of Computer Modern.
% Uncomment the line below if you wish to use the mathtime package:
%\usepackage[cmbold]{mathtime}
% Note that miktex, by default, configures the mathtime package to use commercial fonts
% which you may not have. If you would like to use mathtime but you are seeing error
% messages about missing fonts (mtex.pfb, mtsy.pfb, or rmtmi.pfb) then please see
% the technical support document at http://www.aeaweb.org/templates/technical_support.pdf
% for instructions on fixing this problem.

% Note: you may use either harvard or natbib (but not both) to provide a wider
% variety of citation commands than latex supports natively. See below.

% Uncomment the next line to use the natbib package with bibtex
\usepackage{natbib}

% Uncomment the next line to use the harvard package with bibtex
%\usepackage[abbr]{harvard}

% This command determines the leading (vertical space between lines) in draft mode
% with 1.5 corresponding to "double" spacing.
\draftSpacing{1.5}


% tightlist command for lists without linebreak
\providecommand{\tightlist}{%
  \setlength{\itemsep}{0pt}\setlength{\parskip}{0pt}}




\usepackage{hyperref}

\begin{document}


\title{Replication\_Group\_1}
\shortTitle{A shorter title}
% \author{Author1 and Author2\thanks{Surname1: affiliation1, address1, email1.
% Surname2: affiliation2, address2, email2. Acknowledgements}}


\author{
  Alice Anonymous\\
  Bob Security\thanks{
  Anonymous: Some Institute of
Technology, \href{mailto:alice@example.com}{alice@example.com}.
  Security: Another
University, \href{mailto:bob@example.com}{bob@example.com}.
  Acknowledgements
}
}

\date{\today}
\pubMonth{03}
\pubYear{2023}
\pubVolume{1}
\pubIssue{1}
\JEL{A10, A11}
\Keywords{first keyword, second keyword}

\begin{abstract}
Abstract goes here
\end{abstract}


\maketitle

\section{Introduction}

\begin{itemize}

\item  Briefly explain the original study and its contribution to the literature

\item  Discuss the importance of replication in economics research and the goals of the replication paper

\end{itemize}

\section{Literature Review}

\begin{itemize}

\item Summarize the literature relevant to the original study and the replication

\item Discuss any issues with the original study that have been raised in the literature

\end{itemize}

\section{Methodology}

\begin{itemize}

\item Describe the data and methods used in the original study

\item Explain the steps taken to replicate the original study, including any changes made to the methods or data

\item Discuss any challenges faced in replicating the original study

\end{itemize}

\section{Results}

\begin{itemize}

\item Present the results of the replication study and compare them to the original results

\item Discuss any discrepancies between the two studies and their potential causes

\end{itemize}

\section{Extension}

\begin{itemize}

\item Describe any additional analyses or tests performed using the replicated data

\item Discuss the implications of the extended analysis for the original findings and the literature

\end{itemize}

\section{Conclusion}

\begin{itemize}

\item Summarize the findings of the replication and extension studies

\item Discuss the implications of the replication for the original study and the literature

\item Recommend any changes to the original methods or data for future research

\end{itemize}

\hypertarget{document-styling-guidelines}{%
\subsection{Document Styling
Guidelines}\label{document-styling-guidelines}}

\begin{itemize}
\item Do not use an "Introduction" heading. Begin your introductory material
before the first section heading.

\item Avoid style markup (except sparingly for emphasis).

\item Avoid using explicit vertical or horizontal space.

\item Captions are short and go below figures but above tables.

\item The tablenotes or figurenotes environments may be used below tables
or figures, respectively, as demonstrated below.

\item If you have difficulties with the mathtime package, adjust the package
options appropriately for your platform. If you can't get it to work, just
remove the package or see our technical support document online (please
refer to the author instructions).

\item If you are using an appendix, it goes last, after the bibliography.
Use regular section headings to make the appendix headings.

\item If you are not using an appendix, you may delete the appendix command
and sample appendix section heading.

\item Either the natbib package or the harvard package may be used with bibtex.
To include one of these packages, uncomment the appropriate usepackage command
above. Note: you can't use both packages at once or compile-time errors will result.

\end{itemize}

\hypertarget{section}{%
\subsection{}\label{section}}

\section{First Section in Body}

Sample figure:

\begin{figure}
Figure here.

\caption{Caption for figure below.}
\begin{figurenotes}
Figure notes without optional leadin.
\end{figurenotes}
\begin{figurenotes}[Source]
Figure notes with optional leadin (Source, in this case).
\end{figurenotes}
\end{figure}

Sample table:

\begin{table}
\caption{Caption for table above.}

\begin{tabular}{lll}
& Heading 1 & Heading 2 \\
Row 1 & 1 & 2 \\
Row 2 & 3 & 4%
\end{tabular}
\begin{tablenotes}
Table notes environment without optional leadin.
\end{tablenotes}
\begin{tablenotes}[Source]
Table notes environment with optional leadin (Source, in this case).
\end{tablenotes}
\end{table}

References here (manual or bibTeX). If you are using bibTeX, add your
bib file name in place of BibFile in the bibliography command. \% Remove
or comment out the next two lines if you are not using bibtex.

\bibliographystyle{aea}
\bibliography{references}

\% The appendix command is issued once, prior to all appendices, if any.
\appendix

\section{Mathematical Appendix}


\end{document}
