% AER-Article.tex for AEA last revised 22 June 2011
\documentclass[AER]{AEA}

% The mathtime package uses a Times font instead of Computer Modern.
% Uncomment the line below if you wish to use the mathtime package:
%\usepackage[cmbold]{mathtime}
% Note that miktex, by default, configures the mathtime package to use commercial fonts
% which you may not have. If you would like to use mathtime but you are seeing error
% messages about missing fonts (mtex.pfb, mtsy.pfb, or rmtmi.pfb) then please see
% the technical support document at http://www.aeaweb.org/templates/technical_support.pdf
% for instructions on fixing this problem.

% Note: you may use either harvard or natbib (but not both) to provide a wider
% variety of citation commands than latex supports natively. See below.

% Uncomment the next line to use the natbib package with bibtex
\usepackage{natbib}

% Uncomment the next line to use the harvard package with bibtex
%\usepackage[abbr]{harvard}

% This command determines the leading (vertical space between lines) in draft mode
% with 1.5 corresponding to "double" spacing.
\draftSpacing{1.5}


% tightlist command for lists without linebreak
\providecommand{\tightlist}{%
  \setlength{\itemsep}{0pt}\setlength{\parskip}{0pt}}




\usepackage{hyperref}

\begin{document}


\title{Replication\_Group\_1}
\shortTitle{A shorter title}
% \author{Author1 and Author2\thanks{Surname1: affiliation1, address1, email1.
% Surname2: affiliation2, address2, email2. Acknowledgements}}


\author{
  Alice Anonymous\\
  Bob Security\thanks{
  Anonymous: Some Institute of
Technology, \href{mailto:alice@example.com}{alice@example.com}.
  Security: Another
University, \href{mailto:bob@example.com}{bob@example.com}.
  Acknowledgements
}
}

\date{\today}
\pubMonth{04}
\pubYear{2023}
\pubVolume{1}
\pubIssue{1}
\JEL{A10, A11}
\Keywords{first keyword, second keyword}

\begin{abstract}
Abstract goes here
\end{abstract}


\maketitle

\section{Introduction}

\begin{itemize}

\item  Briefly explain the original study and its contribution to the literature

\item  Discuss the importance of replication in economics research and the goals of the replication paper

\end{itemize}

The study found that low uptake of insurance among smallholder farmers
impairs their ability to manage risks associated with crop failure and
reduce incentives for modernization. Conducted through a randomized
control trial, the study observed a significant increase in insurance
uptake among the sample population by deferring premium payments and
leveraging community leaders to promote the product. The study's
findings have implications for policymakers and stakeholders working to
improve the financial resilience of smallholder farmers in developing
countries.

The study extends the important findings of Casaburi and Willis (2018)
outside the contract farming setting by analyse the uptake increase
under other contracting arrangements, taking into consideration that
most smallholders are not engaged in contract farming. The design also
further extends the findings of Dercon et al.~(2014) because it combines
marketing through informal groups with delayed premium---with possible
synergies both in terms of uptake and defaults. The result is in line
with the findings of Casaburi and Willis (2018) in the context of
contract farming, but they find some evidence that the demand-increasing
effect of the IOU may be larger for people with low savings or income,
supporting the idea that liquidity constraints impede uptake of
insurance.

However, the heterogeneity analysis in this paper only focuses on two
household characteristics and is simply based on subsample regressions,
leaving a gap of heterogeneous effects of the insurance to be filled up.
Based on this, we try to use the causal forests developed by Wager and
Athey (2017) to look deeper into how different types of insurance
designs affect people with diverse socio-economics and production
characteristics in our study, improving the understanding of how to
design contracts for different households in order to achieve better
promotion.

Our replication and extension study aims to build on the original
study's findings by replicating the results using the same data and
methods. Additionally, we will estimate heterogeneous treatment effects
using a multi-arm causal forest approach to investigate whether the
effects of insurance on smallholder farmers vary across different
subgroups of the population. Through this analysis, we hope to
contribute to a better understanding of how insurance can be used to
support agricultural development in low-income countries.

In our study, we aim to estimate heterogeneous treatment effects using a
multi-arm causal forest approach. This differs from the original study,
which also estimated heterogeneous treatment effects but used a
different method. Specifically, our approach allows us to estimate
treatment effects for different subgroups of the population and account
for potential interactions between different variables. Our results will
contribute to a better understanding of the impact of insurance on
smallholder farmers and inform policies aimed at improving their
financial resilience.

(More details of results and conclusions needed.)

The paper is structured into six sections. The first section is the
introduction, which provides an overview of the study and its
objectives. The second section is the literature review, which discusses
the existing research on the topic and identifies the research gaps that
the study aims to address. The third section is the methodology, which
describes the data sources, study design, and statistical techniques
used to analyze the data. The fourth section is the results, which
presents the findings of the study. The fifth section is the extension,
which discusses the multi-arm causal forest approach used to estimate
heterogeneous treatment effects. The sixth and final section is the
conclusion, which summarizes the study's key findings and their
implications for policy and future research.

\section{Literature Review}

\begin{itemize}

\item Summarize the literature relevant to the original study and the replication

\item Discuss any issues with the original study that have been raised in the literature

\end{itemize}

\section{Methodology}

\begin{itemize}

\item Describe the data and methods used in the original study

\item Explain the steps taken to replicate the original study, including any changes made to the methods or data

\item Discuss any challenges faced in replicating the original study

\end{itemize}

C1. Replication Methodology: In this subsection, we describe the
methodology used to replicate the original study. We provide a detailed
explanation of the data and methods used in the original study,
including any necessary assumptions or model specifications. We also
explain the steps taken to ensure that our replication accurately
reproduces the original study's results.

The authors worked together with Oromia Insurance Company (OIC) in
Ethiopia and used multi-level randomization at the \textit{Iddir} level
to assign the 144 \textit{Iddirs} to six experimental arms: 1) Standard
Index Insurance (control group); 2) Standard Index Insurance via
\textit{Iddir} promotions; 3) IOU insurance; 4) IOU insurance with
Contract; 5) IOU insurance via \textit{Iddir} promotions; and 6) IOU
insurance via \textit{Iddir} promotions with Contract. And they
collected data on household demographic characteristics including age,
sex, marital status, education and family size; household income,
households' level of exposure to drought, experience in buying crop
insurance before the experiment, household production and saving
variables.

To verify whether randomisation resulted in balanced groups they regress
household observables and farming observables on treatment group dummies
and a constant. The constant reflects the comparison group and the
coefficients indicate whether other groups are significantly different
from the comparison group. They test for differences between other
groups by Wald tests. The results suggest the randomization worked well.

Then, they present insurance uptake across treatment arms and conduct
regression analysis with and without additional controls including
\textit{Kebele} fixed effects and all baseline socio-economic
characteristics, all showing that uptake change induced by
\textit{Iddir} promotions in isolation is statistically insignificant,
as is IOU with binding contract. They also exclude the subsample from a
certain \textit{Kebele}, Dalota Mati, which all the defaults in the
dataset come from, to increase the statistical power of the analysis,
and the result shows robustness.

They perform a heterogeneity analysis, to figure out if they can
attribute the increase in uptake under IOU insurance to the relaxation
of the liquidity constraint. To proxy for liquidity, they distinguish
between households with above and below-median income, and between
households with and without savings (self-reported). And for both
proxies, the coefficients of the simple IOU product are higher for the
liquidity-constrained. However, while the IOU coefficient of the (more)
constrained subsample is consistently different from zero, and the
coefficient for the complementary sample is not, the relevant
coefficients are not statistically different from each other (according
to a Wald test). This can also be seen by the insignificance of the
coefficients using the interaction term instead of subsamples.

To accurately replicate the results reproduced in the original study, we
follow the steps shown in the paper and try to explain the dicrepancies
if there are any of them. The steps of our replication are described as
follows.

We generate five dummies indicating whether each individual is in one of
the five treatment groups and a dummy indicating the status of uptake.
Then we generate the controls used in the article, including demographic
variables: Age (in years), Sex (male=1; female=0), Marital status
(married=1; not-married=0), Education (years of schooling), Family size,
Total income in the last month (in Birr), Drought (a dummy taking value
of 1 if the household experienced a drought in the last three years),
and Insurance (a dummy taking the value of 1 if the household had
purchased index insurance during the past three years); and farming
variables: capturing quantities of crops produced in the last cropping
season (maize, haricot, teff, sorghum, wheat, and barely), a measure of
total land under cultivation, and a dummy taking the value 1 if the
household had any formal savings.

To conduct balancing tests, we regress observable controls, including
demographic variables and farming variables, on treatment group dummies
and a constant, to see if the coefficients of the group dummies are
statistically significant. The randomisation works well if we see the
coefficients of the treatment dummies are not significant, which shows
that covariates do not affect the treatment assignment and therefore
there is no severe selection bias.

For regression analysis, we regress the uptake status on five group
dummies, and then add controls and \textit{Kebele} fixed effects to the
parsimonious model. Finally, we exclude the subsample from a the
\textit{Kebele} Dalota Mati and run the same regression. We also conduct
Wald tests to test if the coefficients of each two treatment dummies are
statistically the same, verifying whether the treatment effects of
different types of insurance designs differ.

C2. Extension Methodology: In this subsection, we describe the
methodology used for our extension analysis, which seeks to estimate
heterogeneous treatment effects using a multi-arm causal forest
approach. We provide an overview of the statistical methods employed,
including any necessary assumptions or model specifications.

For extension of the replication, we apply a non-parametric causal
forest, which is developed by Wager and Athey (2017) and can achieve
better matching with many covariates, for estimating heterogeneous
treatment effects that extends Breiman's widely used random forest
algorithm. In the potential outcomes framework with unconfoundedness,
causal forests are pointwise consistent for the true treatment effect,
and have an asymptotically Gaussian and centred sampling distribution.
The causal forests give us a better understanding of treatment effect
heterogeneity, so we apply this method to the original paper, which
reveals the treatment effects of a drought insurance, to analyse the
heterogeneity of the effects on individuals with different
characteristics.

The steps of the method can be briefly summarised as follows:

Step 1: The causal forest uses double-sample trees to split the
available training data into two parts: one half (\(I\)) for estimating
the desired response inside each leaf, the other half (\(J\)) for
placing splits. Double-sample regression trees make predictions
\(\hat{\mu}(x)\) using

\begin{equation}
    \hat{\mu}(x)=\frac{1}{|i:X_{i}\in L(x)|}\sum\limits_{|i:X_{i}\in L(x)|}Y_{i}.
\end{equation}

on the leaf containing \(x\), only using the \(I\)-sample observations.
The splitting criteria is the standard for CART regression trees
(minimizing mean-squared error of predictions). Splits are restricted so
that each leaf of the tree must contain \(k\) or more \(I\)-sample
observations. And double-sample causal trees are defined similarly,
except that for prediction we estimate \(\hat{\tau}(x)\) using

\begin{equation}
    \hat{\tau}(x)=\frac{1}{|i:W_{i}=1,X_{i}\in L|}\sum\limits_{|i:W_{i}=1,X_{i}\in L|}Y_{i}-\frac{1}{|i:W_{i}=0,X_{i}\in L|}\sum\limits_{|i:W_{i}=0,X_{i}\in L|}Y_{i}.
\end{equation}

on the \(I\) sample. Following Athey and Imbens (2016), the splits of
the tree are chosen by maximizing the variance of \(\hat{\tau}(x)\) for
\(i\in J\).

Step 2: Propensity trees use only the treatment assignment indicator
\(W_{i}\) to place splits, and save the responses \(W_{i}\) for
estimating \(\tau\). The splits are chosen by optimizing, e.g., the Gini
criterion used by CART for classification.

Step 3: Compute a random forest by Monte Carlo averaging.

Under regularity assumptions, causal forests can realise
unconfoundedness and therefore achieve consistency without needing to
explicitly estimate the propensity. And given all the preliminaries
needed, we can state reliable results on the asymptotic normality of
random forests.

C3. Estimating Heterogeneous Treatment Effects: In this subsection, we
provide a detailed explanation of our approach to estimating
heterogeneous treatment effects using the multi-arm causal forest
method. This includes information on the choice of tuning parameters,
model diagnostics, and methods used for inference.

Our approach to estimating heterogeneous treatment effects using the
multi-arm causal forest method consists of several steps:

Data preparation: We first prepare the data by selecting relevant
covariates and outcome variables, and by converting categorical
variables to numeric ones. We also specify the reference level for the
treatment variable.

Propensity score estimation: We estimate the propensity score, which is
the probability of receiving treatment conditional on the covariates.
This is done using logistic regression, where the treatment variable is
the dependent variable and the covariates are the independent variables.

Causal forest estimation: We then estimate the causal forest using the
multi-arm version of the method, where we include all treatment arms in
the model. The causal forest is a collection of decision trees, where
each tree predicts the outcome variable for a particular treatment arm,
given the covariates and the propensity score.

Treatment effect estimation: Once the causal forest is estimated, we use
it to estimate the heterogeneous treatment effects. This is done by
calculating the difference in predicted outcomes between each treatment
arm and the reference arm, for each individual. These differences are
the estimated treatment effects.

C4. Sensitivity Analyses: In this subsection, we describe the
sensitivity analyses conducted to assess the robustness of our results
to potential model misspecification and data limitations. We also
discuss any limitations or caveats to our approach and provide
suggestions for future research.

We also perform a sensitivity analysis to assess the robustness of our
results to different specifications of the causal forest model. This
involves varying the parameters of the model, such as the number of
trees and the minimum leaf size, and comparing the estimated treatment
effects across different models.

C5. Challenges and Limitations: In this subsection, we discuss any
challenges faced in replicating the original study, including any
discrepancies or issues with the data or code. We also discuss the
limitations of our replication and extension analysis and provide
suggestions for improving future research.

When checking the dataset we obtained from the official resource, we
found that the numbers of \textit{Iddirs} of three groups---IBI, IOU and
IOU\_C, are much larger than the results in the original paper, while
the total number of \textit{Iddirs} and the numbers of observations of
all groups are consistent. This issue indicates the fact that households
in each \textit{Iddir} received different kinds of treatment, which is
inconsistent with what the authors state in the paper. And the original
paper does not explain whether/how they recategorise the households into
new \textit{Iddirs}. We tried to contact the corresponding author, but
he could not provide us a clear explanation in time because of some
private reasons. So, due to this ambiguity, our randomisation does not
work as well as that of the original study, and it leads to regression
results that are different from those obtained by the authors.

\section{Results}

\begin{itemize}

\item Present the results of the replication study and compare them to the original results

\item Discuss any discrepancies between the two studies and their potential causes

\end{itemize}

The results of randomisation are shown in Table 1 and 2, from which we
can see that the significance levels of much more coefficients increase
compared to the original paper. It means that different treatment groups
produce on average very different products and thus the treatment status
may affect almost all important observables. Therefore, we cannot ignore
the selection bias. We believe this is resulted from the data issue we
show in the last section. However, we cannot correct it and the
reliability of our study will be declined.

Table 1 Balance tests on socio-economic variables.

Table 2 Balance tests for production variables and savings.

Figure 2 presents insurance uptake across treatment arms. The delayed
payment of insurance offered by the IOU product increases uptake
substantially when compared to standard insurance, from 8\% to 24\%. By
far, the combination of IOU and promotion through Iddir outperforms all
other treatments: uptake rates increase to around 43\%. Our results are
the same as those in the original paper, except Group 6 that were
offered the most elaborate package including IOU, \textit{Iddir} leader
promotion, as well as the binding contract. The uptake rate of this
group is a bit lower than that of the original experiment (27\%
vs.~32\%). We checked the data and found out the reason for this
discrepancy: there were 80 households (around 5\%) in this group finally
took up an IOU via \textit{Iddir} but WITHOUT a contract, which fills
exactly the gap between our result and the authors'. From this we know
the authors take these households into account when calculating the
uptake rate of this certain group. In our subsequent research, we follow
the authors' way of calculation.

Fig. 2. Uptake rates across IOU treatments, 95\% CI clustered at Iddir
level.

The results of regression analysis (Table 3) are also quite similar to
what the authors obtained. Uptake of individual IOU contracts is not
significantly greater than standard insurance in the presence of a
binding contract. IOU insurance with and without Iddir promotion both
have significant positive effects on the uptake rate. However,
introducing a binding contract to the IOU has a chastening effect on
uptake rates with and without Iddir promotions, suggesting that much of
the additional adoption induced by delayed payment is either motivated
by the prospect of strategic default, or the result of farmers who are
unsure about their ability to pay the future premium -- and thus scared
away once a binding contract is introduced -- also in the absence of
opportunistic intentions. People need a high degree of trust to sign a
binding contract---especially when the consequences of signing are
possibly not fully understood. The Iddir could assuage such concerns by
acting as a trusted third party, resulting in increased uptake rates
also in the presence of a contract. However, in our results, Index
Insurance via Iddir has a significant impact on the uptake in the
parsimonious model, which may also be caused by the data issue.

Table 3 Insurance uptake rates increase under IOU.

\section{Extension}

\begin{itemize}

\item Describe any additional analyses or tests performed using the replicated data

\item Discuss the implications of the extended analysis for the original findings and the literature

\end{itemize}

\section{Conclusion}

\begin{itemize}

\item Summarize the findings of the replication and extension studies

\item Discuss the implications of the replication for the original study and the literature

\item Recommend any changes to the original methods or data for future research

\end{itemize}

In this study, we replicate the main results obtained by Belissa et
al.~(2019) and use the causal forests developed by Waiger and Athey
(2017) to conduct heterogeneity analysis, trying to have a closer look
at the heterogeneous effects of the multi-arm treatments of the
insurance design on households with different socio-economic and
production characteristics.

The replication task does not fully reproduce the results obtained by
the authors. We download the dataset from the official website and find
an obvious data issue that leads to discrepancies in the results---the
numbers of \textit{Iddirs} of three treatment groups are much higher
than those shown in the original paper. This indicates that some
\textit{Iddirs} received different types of policy impacts, which is
different from what the authors state in the paper. With inability to
solve this issue, we show that our randomisation does not work as well
as the original study. However, we still obtain very similar results in
the subsequent analysis, including the uptake rates of different types
of insurance designs and the effects of the insurance designs gained
from the regressions. We can still conclude that delaying weather
insurance payment (IOU) increases uptake and promoting weather insurance
via \textit{Iddir} leaders increases uptake of IOU. And we also find a
negative role of binding contracts.

We then use the multi-arm causal forest to extend the heterogeneity
analysis.

\hypertarget{document-styling-guidelines}{%
\subsection{Document Styling
Guidelines}\label{document-styling-guidelines}}

\begin{itemize}
\item Do not use an "Introduction" heading. Begin your introductory material
before the first section heading.

\item Avoid style markup (except sparingly for emphasis).

\item Avoid using explicit vertical or horizontal space.

\item Captions are short and go below figures but above tables.

\item The tablenotes or figurenotes environments may be used below tables
or figures, respectively, as demonstrated below.

\item If you have difficulties with the mathtime package, adjust the package
options appropriately for your platform. If you can't get it to work, just
remove the package or see our technical support document online (please
refer to the author instructions).

\item If you are using an appendix, it goes last, after the bibliography.
Use regular section headings to make the appendix headings.

\item If you are not using an appendix, you may delete the appendix command
and sample appendix section heading.

\item Either the natbib package or the harvard package may be used with bibtex.
To include one of these packages, uncomment the appropriate usepackage command
above. Note: you can't use both packages at once or compile-time errors will result.

\end{itemize}

\hypertarget{section}{%
\subsection{}\label{section}}

\section{First Section in Body}

Sample figure:

\begin{figure}
Figure here.

\caption{Caption for figure below.}
\begin{figurenotes}
Figure notes without optional leadin.
\end{figurenotes}
\begin{figurenotes}[Source]
Figure notes with optional leadin (Source, in this case).
\end{figurenotes}
\end{figure}

Sample table:

\begin{table}
\caption{Caption for table above.}

\begin{tabular}{lll}
& Heading 1 & Heading 2 \\
Row 1 & 1 & 2 \\
Row 2 & 3 & 4%
\end{tabular}
\begin{tablenotes}
Table notes environment without optional leadin.
\end{tablenotes}
\begin{tablenotes}[Source]
Table notes environment with optional leadin (Source, in this case).
\end{tablenotes}
\end{table}

References here (manual or bibTeX). If you are using bibTeX, add your
bib file name in place of BibFile in the bibliography command. \% Remove
or comment out the next two lines if you are not using bibtex.

\bibliographystyle{aea}
\bibliography{references}

\% The appendix command is issued once, prior to all appendices, if any.
\appendix

\section{Mathematical Appendix}


\end{document}
